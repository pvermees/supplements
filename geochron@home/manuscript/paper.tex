\documentclass[gchron, manuscript]{copernicus}

\usepackage{hyperref,caption}

\begin{document}

\title{FAIR fission track analysis with geochron@home}

\Author[1][p.vermeesch@ucl.ac.uk]{Pieter}{Vermeesch}
\Author[1]{Tim}{Band}
\Author[2]{Jiangping}{He}
\Author[3]{Andrew}{Carter}
\Author[1]{Rex}{Galbraith}

\affil[1]{University College London, London WC1E 6BT, United Kingdom}
\affil[2]{King's College London, London SE1 8WA, United Kingdom}
\affil[3]{Birkbeck, University of London, London WC1E 7HX, United Kingdom}

\runningtitle{FAIR fission track analysis with geochron@home}

\runningauthor{P Vermeesch \emph{et al.}}

\received{}
\pubdiscuss{} %% only important for two-stage journals
\revised{}
\accepted{}
\published{}

\firstpage{1}

\maketitle

\begin{abstract}
Fission track thermochronology is based on the visual analysis of
optical images. This visual process is prone to observer bias. Fission
track datasets are currently reported as small data tables. The
interpretation of these tables requires a high degree of trust between
the fission track analyst and the user of the data. geochron@home is
software that removes this requirement of trust. It combines a
browser-based `virtual microscope' with an online database to provide
FAIR (Findable, Accessible, Interoperable and Reproducible) access to
fission track data.\medskip

geochron@home serves four different purposes. It can be used (1) to
count fission tracks in `private mode', i.e. hidden from other users
on the internet; (2) to archive fission track images and counts for
inspection by other users; (3) to create tutorials for new students of
the fission track method; and (4) to generate crowd-sourced fission
track data, by serving randomly selected selections of images to
citizen scientists. We illustrate these four applications with
examples that demonstrate (1) geochron@home's ability to compare and
combine fission track counts for multiple users within a lab group;
(2) the value of the geochron@home archive in the peer review system;
(3) the use of simple tutorials in teaching novice users how to count
fission tracks; and (4) the `wisdom of crowds' in fission track
identification.\medskip

geochron@home was written in Python and Javascript. Its code is freely
available for inspection and modification, allowing users to set up
their own geochron@home server. Alternatively, users who would like to
upload data to the archive, but do not have the facilities to set up
their own server, may use the server at University College London free
of charge. The archive accepts image stacks acquired on any type of
digital microscope, and accomodates fission track data (counts and
length measurements) from external fission track analysis suites such
as Fission Track Studio and Track\emph{Flow}.\medskip

We anticipate that the introduction of FAIR workflows will make
fission track data more accurate and more future proof. Storing
fission track data online will benefit future developments in fission
track thermochronology. For example, archival datasets of peer
reviewed fission track counts can be used to train and improve machine
learning algorithms for automated fission track analysis. We invite
other geochronological methods to follow the fission track community's
lead in FAIR data processing. This would benefit all the other Earth
Science disciplines that depend on geochronological data.
\end{abstract}

\introduction\label{sec:intro}

Science is on an irreversible trajectory towards greater
openness. Geochronology is no exception to this trend, as this journal
illustrates with its open access and review policies. Funding agencies
increasingly demand that research results and data are shared with the
public. Currently, geochronological data are generally provided as
flat tables of dates or isotopic ratio estimates. However, in other
fields of science such as physics, it is common practice to share the
raw unprocessed measurements along with processing instructions
\citep[e.g.,][]{abbott2016}. This paper moves geochronology in the
same direction. It presents a mechanism to generate and store fully
FAIR \citep[Findable, Accessible, Interoperable and
  Reproducible;][]{wilkinson2016} data in the context of fission track
analysis.\medskip

Unlike most other geochronometers, which require mass spectrometers to
estimate parent-daughter ratios, fission tracks are observed under an
optical microscope and counted by a human observer.  In recent years,
digital microscopy has moved fission track data acquisition from the
objective lens of a microscope to the computer screen
\citep{gleadow2009, vanranst2019, gleadow2019}. Ongoing developments
in artificial intelligence generate further opportunities to improve
the throughput and accuracy of fission track data
\citep{nachtergaele2020}. But despite the richness of the digital
datasets produced by these novel tools, fission track data are still
reported as small summary tables. These tables require an unnecessary
degree of trust between the `producer' and `consumer' of the
data.\medskip

geochron@home is a free and open software platform that allows
geochronologists to share their raw fission track data over the
internet for perusal by peer reviewers, colleagues and the general
public. geochron@home is a virtual petrographic microscope connected to a
database with digital image stacks of etched fission track
samples. The platform can be used to acquire, archive and inspect
fission track data in full adherence to the FAIR data principles. In
Section~\ref{sec:architecture}, we will describe geochron@home's software
architecture in five steps. We will show that this architecture
accommodates imagery from any type of digital microscope. It enables
flexible workflows that can be adapted to four different
applications.\medskip

Using image stacks of Mount Dromedary apatite, we will show how
geochron@home can be used to count fission tracks in `private mode'
(Section~\ref{sec:private}); to archive published fission track
datasets in `public mode' (Section~\ref{sec:GaHa}); to build tutorials
for training purposes (Section~\ref{sec:tutorial}); and to
crowd-source fission track data on the internet
(Section~\ref{sec:crowdsourcing}). Because geochron@home is free and
open, it can be extended and improved by any interested party. We make
some suggestions for future improvements in
Section~\ref{sec:outlook}. We hope that the geochron@home's example
will be followed in other geochronological disciplines, as this will
benefit not only geochronology itself, but all the other disciplines
that depend on it (Section~\ref{sec:conclusions}).

\section{Workflow}\label{sec:architecture}

The geochron@home workflow separates the acquisition of microscope
images from their analysis, providing the flexibility to accommodate
data from different microscope manufacturers. The workflow can be
broken down into five steps.

\begin{enumerate}
\item Acquisition of z-stacks of microscope images in reflected and
  transmitted light for each of the grains in a sample and,
  optionally, for the accompanying external detector
  (Figure~\ref{fig:EDM}). At University College London, this first
  step is currently accomplished by a Python macro within Zeiss' Zen
  Blue software.  However, geochron@home can also accommodate imagery
  from other platforms, such as Fission Track Studio
  \citep[Zeiss;][]{gleadow2009} and Track\emph{Flow}
  \citep[Nikon;][]{vanranst2019}.

\item Prepare the z-stacks for uploading to the geochron@home
  database. This database requires that the images are organised as a
  nested sequence of directories, in which a `project' consists of
  `samples' that comprise a number of `grains'.  Each grain
  corresponds to a numbered sequence of \texttt{.jpeg} images, one for
  each layer of the z-stack. Note that the raw microscope images are
  generally not stored as \texttt{.jpeg} files but in uncompressed
  \texttt{.czi} (for Zen Blue), \texttt{.tif} (for Fission Track
  Studio) or \texttt{.nd2} (for Nikon/Track\emph{Flow})
  formats. Conversion from these raw images to sequences of
  \texttt{.jpeg} files is done with a shell script using Imagemagick
  \citep{still2006}.\medskip

  In addition to the sequence of \texttt{.jpeg} images, a low level
  `grain' folder can also include an optional file called
  \texttt{roi.json} containing the vertices of a default region of
  interest for spontaneous (and/or induced) track counting. The
  database structure can also accommodate existing fission track
  counting results.  For example, if a user has already counted their
  fission tracks in Fission Track Studio, then they can store those
  results in a \texttt{.json} file at the `sample' level
  directory. Because Fission Track Studio stores its results in an
  \texttt{.xml} format, a second conversion script is needed to
  translate those results into an equivalent \texttt{.json} format.

\item Upload the data to the geochron@home platform. geochron@home is
  a Django web-app with a PostgreSQL database.  The database can be
  accessed via a Python API and a (more limited) web-based
  GUI. Accessing the API requires administrator
  privileges. Administrators can create projects, samples and grains;
  download data; and set the access rights of `ordinary' users.
  Projects can be private or public. The source code and installation
  instructions for geochron@home are freely available over GitHub (see
  the Data Availability statement at the end of this paper). This
  allows fission track users to set up their own
  server. Alternatively, fission track labs can upload their data to
  the UCL server by contacting the corresponding author.

\item Analyse the images with a browser-based `virtual fission track
  microscope' powered by the Leaflet library (JavaScript). This
  virtual instrument acts as a front-end to geochron@home. It has a
  simple user interface with controls to zoom, pan and focus in or out
  of the digital image stack. Depending on the permissions granted to
  the user by the administrator, the virtual microsocope offers a
  number of different options.  Entry level `ordinary' users are only
  allowed to count tracks by clicking within a pre-defined `region of
  interest'. In contrast, `superusers' are allowed to define their own
  regions of interest. Once the user is satisfied that they have
  counted all the fission tracks in a particular grain, they can
  submit the results to the server. They are then presented with a new
  set of images until all grains are counted.

\item Post-processing. The fission track data can either be downloaded
  as a flat data table of counts and areas, or as a \texttt{.json}
  file containing the locations of all the counted
  tracks. geochron@home does not provide any tools to post-process
  these files. They are meant to be passed on to other tools such as
  spreadsheet applications or IsoplotR \citep{vermeesch2018c}.
  
\end{enumerate}
  
The five-step workflow can be used for several applications, including
(1) conventional fission track analysis; (2) archiving published
fission track results; (3) building tutorials; and (4) crowd-sourcing
fission track data. The next sections of this paper will illustrate
these applications with real world examples.\medskip

%\begin{figure}[!ht]
{ \centering \includegraphics[width=12cm]{EDM.jpg}
  \captionof{figure}{
    Screenshots of raw fission track data for the external detector
    method in geochron@home. GR: an apatite grain in reflected light;
    GT: the same grain in transmitted light; MR: the corresponding
    mica detector in reflected light; MT: the mica detector in
    transmitted light. White rectangles mark the region of interest
    (ROI), within which an analyst has counted fission tracks by
    marking their etch pits (shown in yellow). The raw data for this
    figure can be viewed on the geochron@home archive
    (\url{https://github.com/pvermees/GaHa}).\medskip}
  \label{fig:EDM}
}%\end{figure}

\section{Counting fission tracks in `private mode'}\label{sec:private}

Administrators can define regions of interest (ROI) and count or edit
the fission track coordinates of any grain in a geochron@home
database. They can also assign other users to groups, and give these
groups access to a subset of the projects in the
database. Administrators have fine control over the permissions of the
groups. For example, they can allow the members of one group to define
their own ROIs, whilst requiring members of another group to count
fission tracks in predefined ROIs.  Groups provide a mechanism to
compare and combine the results of multiple analysts of the same
sample. Figure~\ref{fig:AvP} illustrates this with two sets of fission
track density estimates for the same sample of Mount Dromedary apatite
\citep{green1985b}, analysed by two users (PV and AC).\medskip

Let $N_1$ and $N_2$ be the numbers of spontaneous tracks counted by
the two users over areas $A_1$ and $A_2$, respectively. Then their
estimated track densities are given by $\hat{\rho}_1 = N_1/A_1$ and
$\hat{\rho}_2 = N_2/A_2$. In this situation when the two areas
overlap, $N_1$ and $N_2$ cannot be treated as independent Poisson
counts because the two analysts will count some of the same tracks. It
can be shown that, under simple (ideal) assumptions, the uncertainty
of the ratio of the estimated track densities is given approximately
by
\begin{equation}
  \frac{\mbox{se}(\hat{\rho}_1/\hat{\rho}_2)}{\hat{\rho}_1/\hat{\rho}_2} \approx
  \sqrt{ \frac{1}{N_1} + \frac{1}{N_2} - \frac{2\,N_0}{N_1\,N_2} }
  \label{eq:NO}
\end{equation}

\noindent where $N_0$ is the number of tracks counted by both
observers in the area of overlap ($A_0$, say) between their respective
ROIs.\medskip 

The combined data plots of Figures~\ref{fig:AvP}b and c contain two
sets of counts, with $\sum N_{1} = 686$ and $\sum N_{2} = 679$, and a
weighted mean track density ratio $\hat{\rho}_{1}/\hat{\rho}_{1} =
0.94$ with relative standard error 0.013. PV counted 600 tracks in
those common areas $A_0$, of which 549 were also counted by
AC. Conversely, AC counted 622 tracks of which 549 were also counted
by PV. The ratio of the two analysts' track density estimates based
just on the common area is therefore 600/622 = 0.965 with relative
standard error 0.016, calculated from Equation~\ref{eq:NO}. This is
close to the weighted mean ratio of 0.94 (Figure~\ref{fig:AvP}c), and
is slightly nearer to 1. It indicates that PV under-counts the Mount
Dromedary apatite by 3.5\% relative to AC.  The existence of `observer
bias' is well documented \citep{tamer2025}.  It is one of the reasons
why fission track analysis is often done relative to age standards:
observer bias does not have to be a problem provided that it is
consistent between grains, and between samples.\medskip

%\begin{figure}[!ht]
{ \centering \includegraphics[width=\linewidth]{AvP.pdf}
  \captionof{figure}{Comparison of fission track data for Mount
    Dromedary apatite by two analysts (PV = Pieter Vermeesch and AC =
    Andrew Carter). a) the track counts and ROIs of both users for
    grain 4648, shown in blue (PV) and red (AC); b) comparison of the
    track densities for PV and AC for all 25 grains analysed by the
    two analysts with a reference line for
    $\hat{\rho}_\text{PV}=\hat{\rho}_\text{AC}$; c) radial plot of the
    same data, using Equation~\ref{eq:NO}.}
  \label{fig:AvP}
}%\end{figure}

\section{The geochron@home archive (GaHa)}\label{sec:GaHa}

Once a set of fission track images has been analysed and the analyst
is confident that the results are accurate, the status of the results
can be changed from private to public. This makes the results visible
over the internet as a list of URLs, where each grain number and user
ID corresponds to a unique address. The geochron@home archive
(GaHa) brings fission track geochronology into the era of FAIR
science. It allows peer reviewers to inspect the raw data from which
thermochronological inferences are made. The archive is open to
submissions from any fission track laboratory, free of charge.
However, as mentioned in Section~\ref{sec:architecture}, it is also
possible to establish a new archive elsewhere. At the time of writing,
GaHa contains data for three studies:

\begin{enumerate}
\item{\citet{guo2025}}: this is an LA-ICP-MS based fission track study
  of detrital apatite from the northeastern Tibetan Plateau. It
  contains image stacks of semitracks in 1146 apatite grains from 16
  different samples.
\item{\citet{tamer2025}}: this is a round-robin study in which digital
  image stacks of 44 apatite crystals were circulated among 14
  different analysts. These analysts used the FastTrack image analysis
  software \citep[which is part of the Fission Track Studio
    suite;][]{gleadow2009} to define their own ROIs and count the
  semitracks and horizontally confined fission tracks in them. GaHa
  presents the results of the round-robin experiment as a
  ${44}\times{14}$ grid of URLs.
\item{This study}: All the raw fission track data used in this article
  are available on GaHa, along with the post-processing software that
  was used to produce the figures. Together, these resources provide
  the reader with all the information needed to fully reproduce our
  results, `from cradle to grave'. To our knowledge, this is the first
  geochronological study to do so.
\end{enumerate}

\section{geochron@home tutorials}\label{sec:tutorial}

Given the right permissions (assigned by an administrator), users can
build tutorial pages by annotating features in fission track images.
These features can be tracks or other objects such as scratches,
inclusions, dislocations or holes. A selection of tutorial pages is
presented to new users when they first log into
geochron@home. They must complete the tutorial before being
allowed to count fission tracks.  The tutorial pages can be revisited
at any time by visiting the corresponding link on the
geochron@home landing page.\medskip

The annotations in the current tutorial pages were made by an
experienced fission track analyst (Andrew Carter). This basic tutorial
provides a quick and easy mechanism to train novice users in the art
of fission track analysis. The tutorial pages are in their infancy
and offer only a limited degree of interactivity. Users can click on
features to read the annotations. The tutorial will grow with input
from experienced analysts and future plans include the addition of
fully interactive `quizzes' (Section~\ref{sec:conclusions}). The
limitations of the current tutorials are apparent in the results of
the crowd-sourcing experiment described in the next section of this
paper.

\section{Crowd-sourcing fission track data}\label{sec:crowdsourcing}

%\begin{figure}[!ht]
{ \centering \includegraphics[width=15cm]{radialcrowd.pdf}
  \captionof{figure}{a) Box plots and summary statistics of numbers of
    spontaneous fission tracks counted in each of 25 grains by 68
    novice counters (geology students) all using the same ROIs. Each
    student counted tracks in a subset of the grains. b) The track
    counts for Grain 25 plotted against those for Grain 23 for 37
    students who counted both of those grains. c) Radial plot of the
    ratios of the counts in panel b) using the precisions given by
    Equation~\ref{eq:N12}. In each panel, the counts made by PV are
    added as blue dots.\medskip}
  \label{fig:radialcrowd}
}%\end{figure}

In 1906, Sir Francis Galton visited a county fair in which a contest
was held to guess the weight of an ox. 787 people participated in the
event. Galton discovered that the median of all their estimates was
within 0.8\% of the true weight of the ox and more accurate than 90\%
of the individual estimates. Such is the `wisdom of crowds'
\citep{galton1907a}. Similar effects are seen in fission track
geochronology.\medskip

An interlaboratory comparison study by \citet{miller1985} showed that
the average of several fission track age estimates is closer to the
known age of mineral standards than the age obtained by any individual
observer. Routine measurement of fission track samples by multiple
analysts is prohibitively expensive in a normal laboratory
environment. geochron@home changes this by bringing fission track
analysis to a proverbial `county fair' of citizen scientists.\medskip

Figure~\ref{fig:radialcrowd}a shows box plots and summary statistics
for the raw counts made by 68 students (two of the students were
unable to complete the assignment) for the same selection of 25 grains
that were analysed by AC and PV in Section~\ref{sec:private}. Everyone
counted tracks in the same ROIs, and counts made by PV for those ROIs
have been added as blue dots.  Variation between the mean values is to
be expected because of the differing numbers of tracks due to varying
areas and U contents of the grains. But the variation between counts
within grains (shown by the standard deviations) is due entirely to
differences between the students' recognising and counting exactly the
same tracks. These standard deviations increase with the mean number
of tracks and are considerable in size, being on average about 40\% of
the mean.  Also, the vast majority of students counted fewer tracks
than PV did, often many fewer. In 21 of the 25 grains, PV's count is
above the upper quartile of the students', and on two occasions
someone counted no tracks at all.  With expert trained counters one
would expect much smaller differences between counts.  Nevertheless
there are many repeated counts also.\medskip

Figure~\ref{fig:radialcrowd}b shows a scatter plot of pairs of counts
for two grains made by 37 students who counted both of them. They
average around 29 and 30 tracks per grain compared with PV's pair of
52 and 50, which is highlighted in blue. There is a positive
correlation between the pairs, consistent with systematic observer
effects (i.e., people who count a low value in one grain tend to count
similarly low in the other, and vice versa).  Comparing other pairs of
grains shows similar results, with correlations varying between 0.3
and 0.9. In all cases we have looked at, the residual variation, after
allowing for systematic differences between students, is consistent
with Poisson variation.  Figure~\ref{fig:radialcrowd}c shows a radial
plot of the ratios of the counts in Panel~b, using the precisions
given by
\begin{equation}
  \frac{\mbox{se}(N_1/N_2)}{N_1/N_2}
  \approx
  \sqrt{
    \frac{1}{N_1} + \frac{1}{N_2}
  }
  \label{eq:N12}
\end{equation}

The context here is rather different from that in
Figure~\ref{fig:AvP}b, where we were comparing the ratios of counts
made by two analysts on each of 25 grains. Here we are comparing the
ratios of counts for the same two grains made by each of 37
analysts. There is no theoretical justification for
Equation~\ref{eq:N12} here, but it still provides a useful benchmark
for assessing variation between the ratios. In
Figure~\ref{fig:radialcrowd}c, that variation is consistent with
Poisson variation (as it is for most other pairs of grains) but with
expert counters one would expect less variation than that. However, in
spite of the under-counting by the students, the pooled students'
ratio of 0.76 is close to PV's ratio of 0.70. This happens quite
often in this data set, but not always, and there are some pairs of
grains where the students' pooled ratio differs somewhat from
PV's.\medskip

As mentioned in Section~\ref{sec:architecture}, geochron@home stores
the actual track positions marked by the users. This raw data can be
downloaded as a \texttt{.json} file and inspected in detail.
Figure~\ref{fig:crowdsourcing} is a two-dimensional histogram for the
x-y positions of all the track positions generated by the students in
the two apatite crystals. Visual comparison of the histogram with the
optical image confirms that the students unanimously identified the
most obvious semitracks, which contain a clearly visible etch pit and
tail. Shorter and fainter tracks received fewer clicks. Datasets like
this can be used to replace integer counts of fission tracks with
probabilities, reflecting the ambiguity of some fission track
datasets.\medskip

Figure~\ref{fig:crowdsourcing} also shows that some students counted
the tails of the fission tracks rather than their etch pits, despite
being told the opposite in the tutorial. Fixing this issue will
require some improvements to the tutorial pages
(Section~\ref{sec:outlook}).\medskip

%\begin{figure}[!ht]
{  \centering
  \includegraphics[width=9cm]{4649vs4673.pdf}
  \captionof{figure}{
    a,b) Optical images in transmitted light of 
    Mount Dromedary apatites 4649 and 4673 in the crowd-sourcing experiment.
    c,d) Two-dimensional histogram of the track locations for the
    two grains, as identified by the citizen scientists.
    \medskip}
  \label{fig:crowdsourcing}
}%\end{figure}

\section{Outlook}\label{sec:outlook}

geochron@home has been in development for a decade and remains a work
in progress. Planned improvements include:

\begin{enumerate}
\item Interactive tutorial pages. To ensure that novice users do not
  count the tails but the etch pits of fission tracks, we will add a
  `quiz' to the tutorial pages. Only users who click on the `correct'
  features in an unlabelled set of images will be allowed to count new
  samples. Section~\ref{sec:crowdsourcing} shows that, despite the
  large scatter of the raw track counts obtained by the citizen
  scientists, their pairwise ratio estimates scatter evenly around the
  expert opinion. This suggests that the collective wisdom of the
  entire group of students is greater than that of the individual
  analysts. When applied to a group that has achieved a good level of
  training and experience, crowd sourcing should improve the precision
  and accuracy of fission track data.
\item Length measurements. The geochron@home archive already
  contains horizontally confined fission track measurements
  \citep{tamer2025}.  However, these results must be generated
  externally (e.g., using Fission Track Studio) and uploaded via a
  \texttt{.json} file.  The virtual microscope curently lacks the
  functionality to generate length data within
  geochron@home. This functionality will be added in a future
  update.
\item Dpar and Dper. Etch pits are currently stored as simple sets of
  x- and y-coordinates.  In reality, etch pits have a finite length
  (`Dpar') and width (`Dper'), which serve as useful indicators for
  the horizontal etch rates along the c-axis and parallel to it
  \citep{donelick1993}. Functions will be added to measure and
  visualise this type of data in geochron@home.
\item Machine learning. Data science is experiencing an artificial
  intelligence revolution that has already started to transform the
  fission track method \citep{nachtergaele2020}.  Convolutional neural
  networks must be trained with example data. geochron@home
  is ideally suited for this task. Section~\ref{sec:crowdsourcing}
  showed how the collective wisdom of multiple fission track analysts
  can label fission track images with probabilities rather than
  counts. This data format is close to the form in which data are
  treated within an AI algorithm.\medskip

  Once trained on historical data, AI algorithms can be used to count
  fission tracks automatically. Following the model of Fission Track
  Studio \citep{gleadow2009, gleadow2019}, machine learning can be
  used to reverse the fission track counting process. Instead of
  asking users to count the fission tracks in a sample, the software
  can ask them to check the results proposed by an AI algorithm, and
  to remove any features that are \emph{not} fission tracks.\medskip

  Regardless of whether fission tracks were counted manually or with a
  machine, the value of the geochron@home archive remains the
  same. It is important to document data so that samples can be
  reanalysed in the future, for example when a new and improved
  generation of machine learning algorithms becomes available.
\end{enumerate}

These four improvements will be made by ourselves pending additional
funding. However, because geochron@home is free and open software, we
invite any interested parties to join the effort and extend or improve
our code.

\section{Conclusions}\label{sec:conclusions}

This paper introduced geochron@home, a software platform for FAIR
fission track analysis. We demonstrated four different applications of
this platform using real data. Putting the FAIR data paradigm into
practice, all the imagery, counts and source code for this paper are
publicly available. Using these resources, the reader can reproduce
all the results that were presented in this publication.  We encourage
other geochronologists to follow this example.  FAIR data promises to
address the reproducibility crisis in science
\citep{miyakawa2020}.\medskip

geochron@home's rich archive of raw data can be reanalysed in
the future. We anticipate that the adoption of FAIR data processing
workflows will open up new research opportunities. For example,
archived pairs of peer-reviewed fission track images and counts could
be used to train the next generation of automated machine learning
algorithms. Conversely, it is also possible that future improvements
in fission track images analysis will be used to update the count data
for published datasets, improving their accuracy.\medskip

Another advantage of the geochron@home wokflow is the separation of
image acquisition and image analysis. This separation reduces the
hardware requirements for fission track geochronology. It opens up the
possibility to share resources. State-of-the-art digital microscopes
are expensive. Using geochron@home, a single microscope can serve
multiple users and make fission track analysis more
affordable.\medskip

The fission track method has always been a test bed for new
geochronological developments.  Because fission track data are
imprecise, the fission track community has sollicited the help of
statisticians and mathematicians to develop its analytical protocols.
Other geochronological communities are still catching up with concepts
and tools such as overdispersion, mixture modelling and radial plots,
which have been commonplace in fission track analysis for decades
\citep{vermeesch2019b}. In a similar vein, the subjective nature of
fission track identification has prompted the fission track community
to organise inter-laboratory comparisons and round-robin studies long
before other geochronological communities
\citep{miller1985,tamer2025}.\medskip

With the development of geochron@home, fission track thermochronology
is once again ahead of the pack in terms of FAIR data analysis.
geochron@home currently only stores images and counts. This is enough
to reproduce the results of fission track studies using the external
detector method, but not for LA-ICP-MS based data. FAIR data
processing of LA-ICP-MS data requires a new generation of mass
spectrometer data reduction software. We are currently working on this
\citep{vermeesch2025b}. The development of FAIR ICP-MS data pipelines
will not only benefit fission track analysis but other chronometers as
well, such as in-situ U--Pb, Rb--Sr and Lu--Hf.\medskip

With the establishment of FAIR data, geochronology will be well placed
to avoid the reproducibility problems that have plagued other fields
of science.

\codedataavailability{geochron@home is free software released under
  the GPL-3 license. The package and its source code are available
  from \url{https://github.com/pvermees/geochron-at-home} (last
  access: July~21, 2025).  The raw data (imagery) are available at the
  geochron@home archive (\url{https://github.com/pvermees/GaHa}, last
  access: July~21, 2025).  R-scripts to reproduce the figures are
  provided in the supplementary information
  (\url{https://github.com/pvermees/supplements}, last access:
  July~21, 2025).}

%\appendix
%\section{}

%\appendix{}

%\noappendix

\authorcontribution{Pieter Vermeesch designed the study, acquired the
  funding, counted fission tracks and wrote the paper. Jiangping He
  created geochron@home. Tim Band expanded geochron@home and wrote the
  accompanying microscope image acquisition software. Andrew Carter
  provided the samples, prepared the training data and counted fission
  tracks. Rex Galbraith derived Equation~\ref{eq:NO} and verified all
  the other calculations.}

\competinginterests{PV is an Associate Editor of \emph{Geochronology}.}

\begin{acknowledgements}
This research has been supported by the Natural Environment Research
Council (grant no. NE/T001518/1), awarded to Pieter Vermeesch. We would like
to thank the students of GEOL0017 (`Isotope Geology') for their contribution
to the crowd-sourcing experiment of Section~\ref{sec:crowdsourcing}.
\end{acknowledgements}

\bibliographystyle{copernicus}
\bibliography{biblio.bib}

\end{document}
